%!TEX program = xelatex
\documentclass[
	lang=cn,
	color=blue
]{elegantbook}
\usepackage{tabularx,booktabs} 
\usepackage{listings}
\usepackage{xr}
\externaldocument{../problem/problem}

\newcolumntype{M}{>{\centering\arraybackslash}X}

\begin{document}

\begin{titlepage}
    \begin{center}
		\LARGE
		\textbf{ACM算法与微应用开发实验室21届成员选拔赛题解} \par
		\normalsize
		\vspace{0.5cm}
		2021年11月6日
	\end{center}

    \section*{题目概览}
    \begin{center}
        \begin{tabularx}{\textwidth}{M|M|M|M}
            \toprule
            \textbf{题目编号} & \textbf{题目名称} &\textbf{出题人} & \textbf{做法} \\
            \midrule
            A & \ref*{pro:1} & \ref*{aut:1} &线性筛 \\
            B & \ref*{pro:2} & \ref*{aut:2} &搜索 \\
            C & \ref*{pro:3} & \ref*{aut:3} &贪心 \\
            D & \ref*{pro:4} & \ref*{aut:4} &前缀和 \\
            E & \ref*{pro:5} & \ref*{aut:5} &模拟 \\
            F & \ref*{pro:6} & \ref*{aut:6} &模拟\\
            \bottomrule
        \end{tabularx}
    \end{center}

    \section*{鸣谢}
    感谢Tifa大佬为本次比赛贡献的题目

    \paragraph*{注意:因篇幅显示,部分标程省略了头文件部分。}

\end{titlepage}

\chapter*{A.\quad \ref*{pro:1}}
\section*{做法}

\section*{标程}

\chapter*{B.\quad \ref*{pro:2}}
\section*{做法}

\section*{标程}

\chapter*{C.\quad \ref*{pro:3}}
\section*{做法}

\section*{标程}

\chapter*{D.\quad \ref*{pro:4}}
\section*{做法}

\section*{标程}

\chapter*{E.\quad \ref*{pro:5}}
\section*{做法}

\section*{标程}

\chapter*{F.\quad \ref*{pro:6}}
\section*{做法}
签到题,题意即求出$\lfloor A \times (1-p1\%) \times (85\%)^0 \rfloor + \lfloor A \times (1-p2\%) \times (85\%)^1 \rfloor + \lfloor A \times (1-p3\%) \times (85\%)^2 \rfloor + \lfloor A \times (1-p4\%) \times (85\%)^3 \rfloor$的值

可以使用秦九韶算法,也可以\lstinline{cout}一行搞定。

\section*{标程}
\lstinline{cout}一行搞定:
\begin{lstlisting}
int main()
{
    int t;
    cin>>t;
    while(t--)
    {
        int A,p1,p2,p3,p4;
        cin>>A>>p1>>p2>>p3>>p4;
        cout<<((int)(A*(1-p1/100.0)) + (int)(A * 0.85 * (1-p2/100.0)) + (int)(A * 0.85 * 0.85 * (1-p3/100.0)) + (int)(A * 0.85 * 0.85 * 0.85 * (1-p4/100.0)))<<endl;
    }
    return 0;
}
\end{lstlisting}

秦九韶算法:
\begin{lstlisting}
int p[7];
int main()
{
    int t;
    cin>>t;
    while(t--)
    {
        int A;
        cin>>A;
        for(int i=1;i<=4;i++) cin>>p[i];
        double mul = 1;
        int sum = 0;
        for(int i=1;i<=4;i++, mul *= 0.85)
            sum += (int)(A * mul * (1 - p[i] / 100.0));
        cout<<sum<<endl;
    }
    return 0;
}
\end{lstlisting}

\end{document}