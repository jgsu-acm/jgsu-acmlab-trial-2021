\documentclass{../cpct/ctsol}

\title
{
    ACM 算法与微应用开发实验室 21 届成员选拔赛\par
    暨 2021 年 10 月月赛题解
}
\date{2021 年 11 月 6 日}

\begin{document}

\maketitle
\addsolution{K 素数筛}{Tifa}{线性筛}
\addsolution{天马行空}{AgOH}{搜索}
\addsolution{双端队列}{AgOH}{贪心}
\addsolution{水的体积}{AgOH}{前缀和}
\addsolution{棋牌室}{Tifa}{模拟}
\addsolution{倍思亲}{AgOH}{模拟}

\section*{题目概览}

\solutiontab

\section*{鸣谢}

感谢 Tifa 大佬为本次比赛贡献的题目。

\makesolution
\section*{做法}

本题考查对线性筛算法的理解。

我们回想一下线性筛在干什么,首先我们有一个 \verb|visit| 数组用来存我们是否已经筛过当前的数,还有个 \verb|prime| 数组记录我们筛出的素数。我们从2开始遍历,如果当前的数没被筛掉,那它就是素数。同时对于我们枚举的每个数 $i$,把 $i$ 与素数 $p_j$ 的乘积筛掉,也就是在 visit 数组打上标记。$p_j$ 要满足 $p_j<p$, 其中 $p$ 是 $i$ 的最小素因子,这样能确保每个数都会被不重不漏地访问一遍。

那我们怎么找 $k-$素数呢?

我们再定义一个数组 \verb|k_cnt[]|, \verb|k_cnt[i]| 表示 i 能表示为多少个素数的乘积。显然,对所有的素数 $p$, \verb|k_cnt[p]| 都应赋为 1。并且 \verb|k_cnt[i * p] = k_cnt[i] + 1|。

所以我们只需要在标记 $i\times p_j$ 时把 \verb|k_cnt[i * prime[j]]| 赋为 \verb|k_cnt[i] + 1| 即可。

完全 $k-$素数同理。

\section*{标程}

\std{A}

\makesolution
\section*{做法}

一道不能再水的搜索,数据规模很小,dfs 或者 bfs 都可以。

\section*{dfs}

\std{B1}

\section*{bfs}

\std{B2}

\makesolution
\section*{做法}

因为只有弹出操作,所以本题并不是考察双端队列的用法,只是起了这样一个名字而已……

例如对于题目样例给出的双端队列 $2,1,5,4,3$,第一次弹出应该弹出什么?显然应该弹出 $2$ 而不是 $3$,因为若弹出 $3$ 则再也不能弹出 $2$ 了,但弹出 $2$ 后还可以弹出 $3$。

所以本题利用贪心思想即可解决:在两侧都能弹出的时候弹出数字较小的一侧的数字;否则哪边能弹出就弹出哪边;当两边都不能弹出或队列为空时输出答案即可。

因为题目所给数据只是 $1 \sim n$ 的一个排列(数字不会重复),所以难度较低。本题也可以扩展到数字可以重复的情况,留作思考题。

\section*{标程}

\std{C}

\makesolution
\section*{做法}

需要使用一种叫做“前缀和”的简单算法。

注意到:

$$a_l+a_{l+1}+ \cdots + a_r = (a_1 + a_2 + \cdots + a_r) - (a_1 + a_2 + \cdots + a_{l-1})$$

我们定义:

$$pre_i = a_1 + a_2 + \cdots + a_i$$

于是:

$$a_l+a_{l+1}+ \cdots + a_r = pre_r - pre_{l-1}$$

所以对于每次 $[l,r]$ 查询,只需要 $O(1)$ 的时间复杂度即可得出结果,而 $pre_i$ 可以通过 $pre_i = pre_{i-1} + a_i$ 的递推式从而在 $O(n)$ 的时间复杂度内推出来。总时间复杂度为 $O(n+m)$。

\section*{标程}

\std{D}

\makesolution
\section*{做法}

一个模拟题,写法很多。这里给出其中一种。

我们考虑枚举出雀头,删去雀头的两张牌后判断剩余的 $12$ 张牌是否为 $4$ 个面子。

我们可以直接对每种牌模 $3$,之后判断剩下的牌是不是顺子即可。

\section*{标程}

\std{E}

\makesolution
\section*{做法}

签到题,题意即求出下式的值:

$$\lfloor A \times (1-p1 \%) \times (85 \%)^0 \rfloor + \lfloor A \times (1-p2 \%) \times (85 \%)^1 \rfloor + \lfloor A \times (1-p3 \%) \times (85 \%)^2 \rfloor + \lfloor A \times (1-p4 \%) \times (85 \%)^3 \rfloor$$

可以使用秦九韶算法,也可以直接输出上式的结果。

\section*{直接输出}

\std{F1}

\section*{秦九韶算法}

\std{F2}

\end{document}
