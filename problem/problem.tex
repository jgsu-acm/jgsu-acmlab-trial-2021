%!TEX program = xelatex
\documentclass[
	lang=cn,
	color=green
]{elegantbook}
\usepackage{tabularx,booktabs} 
\usepackage{graphicx}
\usepackage{listing}
\usepackage{pdfcomment,bookmark}
\usepackage{tikz}

\title{ACM算法与微应用开发实验室21届成员选拔赛题目}
\date{2021年11月6日}
\makeatletter
\renewcommand{\maketitle}{
	\begin{center}
		\LARGE
		\textbf{\@title} \par
		\normalsize
		\vspace{0.5cm}
		\@date
	\end{center}
}

% copy from https://tex.stackexchange.com/questions/271062/labeling-a-text-and-referencing-it-later
\newcommand{\labeltext}[2]{
  \@bsphack
  \csname phantomsection\endcsname
  \def\@currentlabel{#1}{\label{#2}}
  \@esphack
}
\makeatother

\newcolumntype{M}{>{\centering\arraybackslash}X}

\newcommand{\problemtab}[2]{
	\begin{tabularx}{#2}{M|M|M|M|M|M}
		\toprule
		\textbf{题目编号} & \textbf{题目名称} & \textbf{运行时间上限} & \textbf{运行内存上限} & \textbf{题目类型} & \textbf{命题人} \\
		\midrule
	\end{tabularx}

	\foreach \i in #1{
		\begin{tabularx}{#2}{M|M|M|M|M|M}
			\i & \ref*{pro:\i} & \ref*{tim:\i} & \ref*{mem:\i} & \ref*{typ:\i} & \ref*{aut:\i} \\
		\end{tabularx}

	}

	\begin{tabularx}{#2}{M|M|M|M|M|M}
		\toprule
	\end{tabularx}
}

\newcommand{\addproblem}[6]{
	\labeltext{#2}{pro:#1}
	\labeltext{#3ms}{tim:#1}
	\labeltext{#4M}{mem:#1}
	\labeltext{#5}{typ:#1}
	\labeltext{#6}{aut:#1}
}

\newcommand{\problemheader}[1]{
	\chapter*{#1.\quad \ref*{pro:#1}}
	\addcontentsline{toc}{chapter}{A.\ref*{pro:#1}}
	\begin{center}
		运行时间上限:\ref*{tim:#1} \quad 运行内存上限:\ref*{mem:#1} \quad 题目类型:\ref*{typ:#1} \quad 命题人:\ref*{aut:#1}
	\end{center}
}

\newcommand{\problembody}[4]{
	\section*{题目描述}
	#1
	\section*{输入格式}
	#2
	\section*{输出格式}
	#3
	\section*{输入输出样例}
	#4
}

\newcommand{\makeproblem}[5]{
	\problemheader{#1}
	\problembody{#2}{#3}{#4}{#5}
}

\begin{document}
% 题目信息
\addproblem{A}{K素数筛}{3000}{256}{传统}{Tifa}
\addproblem{B}{天马行空}{1000}{256}{传统}{AgOH}
\addproblem{C}{双端队列}{1000}{256}{传统}{AgOH}
\addproblem{D}{水的体积}{1000}{256}{传统}{AgOH}
\addproblem{E}{棋牌室}{1000}{256}{传统}{Tifa}
\addproblem{F}{倍思亲}{1000}{256}{传统}{AgOH}

\begin{titlepage}
	\maketitle
	\section*{比赛信息}
	\begin{tabularx}{450pt}{M|M|M|M}
		\toprule
		\textbf{赛制}       & \textbf{语言}       & \textbf{时长} & \textbf{题目数量} \\
		\midrule
		ACM\ |个人赛|不封榜 & C/C++,Python,Java & 3小时         & 6                 \\
		\bottomrule
	\end{tabularx}

	\section*{题目概况}
	\problemtab{{A,B,C,D,E,F}}{450pt}

	\section*{编译命令}
	\small
	\begin{tabularx}{450pt}{l|X}
		\toprule
		\verb|C(gcc 9.4)|  & \verb|gcc -DONLINE_JUDGE -O2 -w -std=c11 {src_path} -lm -o {exe_path}|  \\
		\verb|C++(g++ 9.4)|  & \verb|g++ -DONLINE_JUDGE -O2 -w -std=c++14 {src_path} -lm -o {exe_path}|  \\
		\verb|Java(OpenJDK 11)|  & \verb|javac {src_path} -d {exe_path} -encoding UTF8|  \\
		\verb|Python2.7|  & \verb|python -m py_compile {src_path}|  \\
		\verb|Python3.6| & \verb|python3 -m py_compile {src_path}| \\
		\bottomrule
	\end{tabularx}
	\normalsize

	\section*{注意事项}
	\begin{itemize}
		\item C/C++中函数main()的返回值类型必须是int,程序正常结束时的返回值必须是0。
		\item C/C++代码必须完全符合GNU C/C++ 标准,不能使用诸如绘图、Win32API、中断调用、硬件操作或与操作系统相关的API。
		\item C/C++代码中允许使用STL类库。
	\end{itemize}

	\paragraph*{} 祝大家取得好成绩!

\end{titlepage}

\makeproblem{A} {
	Tifa: 对给定的$n$, 如何快速地在$[1,n]$中筛出所有的素数?

	AgOH: 这不有手就行? 看我表演一波线性筛素数...

	Tifa: 那如果将恰好能表示为$k$个素数乘积的数称为$k$-素数, 对给定的$n,k$,如何快速地在$[1,n]$中筛出所有的$k$-素数?

	AgOH: ???

	Tifa: 那如果将恰好能表示为$k$个\textbf{不同}素数乘积的数称为 完全$k$-素数, 对给定的$n,k$,如何快速地在$[1,n]$中筛出所有的完全$k$-素数?

	AgOH: ??????……不慌,实验室的成员们应该能帮我解决这个问题,我先去问问他们去(

}{
	多组数据。

	第一行为一个整数$t~(1\leq t\leq 10^3)$, 表示数据组数。

	接下来$t$行每行有两个整数$n~(1\leq n\leq 2\times 10^6)$,$k~(0\leq k\leq n)$, 含义同描述。

}{
	每组数据输出两行, 其中第一行输出$k$-素数的结果,第二行输出完全$k$-素数的结果。

	每行首先输出一个数$m$,表示$[1,n]$中所有$k$-素数/完全$k$-素数的个数。

	若$m$为$0$,则结束该行的输出。

	若$m$不为$0$,则接下来隔一个空格后输出一个数$x$, 为$[1,n]$中所有$k$-素数/完全$k$-素数的\textbf{异或和}。

}{
	\begin{tabularx}{450pt}{X|X}
		\toprule
		输入样例 & 输出样例 \\
		\midrule
		4        & 0        \\
		1 1      & 0        \\
		20 1     & 8 7      \\
		10 2     & 8 7      \\
		30 3     & 4 1      \\
		         & 2 12     \\
		         & 7 27     \\
		         & 1 30     \\
		\bottomrule
	\end{tabularx}

}
\clearpage

\makeproblem{B}{
	弈缘棋社每年一度的新生赛马上就要开始了!作为弈缘棋社象棋水平最低的棋手的AgOH为了不暴露自己的真实实力,设置了一个问题,只有答出这个问题的人才能在象棋盘上吊打AgOH一番。问题如下:

	给定中国象棋棋盘上的两个点$p1(x1,y1),p2(x2,y2)$,马从$p1$走到$p2$最少需要多少步呢?

	作为参赛棋手的小T非常想要赢AgOH一盘棋,可他实在做不出毒瘤AgOH出的这道题目,于是他马上来求助你,你能帮帮他吗?

	注:中国象棋中棋子“马”的行动规则(不考虑蹩马腿):每步一直一斜,即先横着或直着走一格,然后再斜着走一个对角线,俗称“马走日”。棋子不能走出棋盘。如图所示的$8$个绿点为$(4,7)$位置上的马的行动范围。

	\begin{center}
		\includegraphics[scale=0.1]{images/chess.jpg}
	\end{center}

}{
	一行,四个整数$x_1,y_1,x_2,y_2(1 \leq x_1,x_2 \leq 10; 1 \leq y_1,y_2 \leq 9)$

}{
	一行,一个整数:若马能从$p1$走到$p2$,输出马从$p1$走到$p2$最少需要多少步;否则输出$-1$。

}{
	\begin{tabularx}{450pt}{X|X}
		\toprule
		输入样例 & 输出样例 \\
		\midrule
		1 1 2 3  & 1        \\
		\bottomrule
	\end{tabularx}
	\vspace{0.5cm}

	\begin{tabularx}{450pt}{X|X}
		\toprule
		输入样例 & 输出样例 \\
		\midrule
		5 5 5 5  & 0        \\
		\bottomrule
	\end{tabularx}
	\vspace{0.5cm}

	\begin{tabularx}{450pt}{X|X}
		\toprule
		输入样例 & 输出样例 \\
		\midrule
		8 5 10 8 & 3        \\
		\bottomrule
	\end{tabularx}

}
\clearpage

\makeproblem{C}{
	双端队列是一种常用的数据结构,其支持四种操作:在队头插入一个元素、在队尾插入一个元素、弹出队头的元素、弹出队尾的元素。(弹出:即将元素从双端队列中拿走)

	给你一个存放有$1 \sim n$的\textbf{一个排列}的双端队列,你可以对其进行若干次弹出操作(可以弹出队头或队尾),且弹出的元素按弹出顺序排列出的数列必须为递增的,请问最多可以弹出多少个元素?

	例如:如下双端队列最多可以弹出$4$个元素,操作顺序为弹出队首元素$2$、弹出队尾元素$3$、弹出队尾元素$4$、弹出队尾元素$5$。

	\begin{center}
		\begin{tabularx}{200pt}{M|M|M|M|M}
			\toprule
			2 & 1 & 5 & 4 & 3 \\
			\bottomrule
		\end{tabularx}
	\end{center}

}{
	第一行,一个整数,$n(1 \leq n \leq 5 \times 10^5)$。

	第二行,$n$个整数,代表初始的双端队列。

}{
	一行,一个整数:最多能弹出多少个数字。

}{
	\begin{tabularx}{450pt}{X|X}
		\toprule
		输入样例  & 输出样例 \\
		\midrule
		5         & 4        \\
		2 1 5 4 3 &          \\
		\bottomrule
	\end{tabularx}
	\vspace{0.5cm}

	\begin{tabularx}{450pt}{X|X}
		\toprule
		输入样例      & 输出样例 \\
		\midrule
		7             & 7        \\
		1 3 5 6 7 4 2 &          \\
		\bottomrule
	\end{tabularx}
	\vspace{0.5cm}

	\begin{tabularx}{450pt}{X|X}
		\toprule
		输入样例 & 输出样例 \\
		\midrule
		3        & 3        \\
		1 2 3    &          \\
		\bottomrule
	\end{tabularx}
	\vspace{0.5cm}

	\begin{tabularx}{450pt}{X|X}
		\toprule
		输入样例 & 输出样例 \\
		\midrule
		4        & 4        \\
		1 2 4 3  &          \\
		\bottomrule
	\end{tabularx}

}
\clearpage

\makeproblem{D}{
	有$n$个矿泉水瓶,第$i$个瓶子里装有$a_i\ ml$水。

	AgOH会对你进行$m$次询问,每次询问一个区间$[l,r]$,请你告诉AgOH第$l$个瓶子到第$r$瓶子之间(包含第$l$个和第$r$个)的所有瓶子里共装有多少$ml$的水。

}{
	第一行,两个整数$n,m(1 \leq n,m \leq 10^5)$。

	第二行,$n$个整数$a_i(1 \leq a_i \leq 550)$。

	第三到第$m+1$行:每行两个整数$l,r(1 \leq l \leq r \leq n)$。

}{
	共$m$行,第$i$行代表第$i$次询问的答案。

}{
	\begin{tabularx}{450pt}{X|X}
		\toprule
		输入样例  & 输出样例 \\
		\midrule
		5 3       & 3        \\
		1 2 3 4 5 & 9        \\
		1 2       & 15       \\
		2 4       &          \\
		1 5       &          \\
		\bottomrule
	\end{tabularx}

}
\clearpage

\makeproblem{E}{
	本次比赛实验室内座位布局方式参考……打麻将。

	既然都坐成了棋牌室的样子了,不打打麻将实在是说不过去。

	当然比赛过程中打一局完整的麻将是不现实的,你只需要判断一副手牌是否是和牌即可。

	\begin{remark}
		\hspace{0.15cm} \textbf{和牌规则}:对于手中的14张手牌,和牌需达成以下条件:

		\begin{itemize}
			\item 有一个 \textbf{雀头}。
			\item 有四个 \textbf{面子}。
			\item 雀头和各个面子间没有交叉的牌
		\end{itemize}

		\hspace{0.15cm} \textbf{雀头}:两张同花色的牌,如 \includegraphics[scale=0.5]{images/mahjong/1s.png}\includegraphics[scale=0.5]{images/mahjong/1s.png}

		\hspace{0.15cm} \textbf{面子}:包括 \textbf{刻子} 和 \textbf{顺子} (不要问为啥没杠,莫抬杠)

		\begin{itemize}
			\item \textbf{刻子}:三张同花色的牌,如 \includegraphics[scale=0.5]{images/mahjong/1m.png}\includegraphics[scale=0.5]{images/mahjong/1m.png}\includegraphics[scale=0.5]{images/mahjong/1m.png}。
			\item \textbf{顺子}:三张相邻的同类型牌(只包括条/索、饼/筒、万三种),如 \includegraphics[scale=0.5]{images/mahjong/1m.png}\includegraphics[scale=0.5]{images/mahjong/2m.png}\includegraphics[scale=0.5]{images/mahjong/3m.png}。

			      \includegraphics[scale=0.5]{images/mahjong/9m.png} 与 \includegraphics[scale=0.5]{images/mahjong/1m.png}、\includegraphics[scale=0.5]{images/mahjong/9p.png} 与 \includegraphics[scale=0.5]{images/mahjong/1p.png}、\includegraphics[scale=0.5]{images/mahjong/9s.png} 与 \includegraphics[scale=0.5]{images/mahjong/1s.png} 不算作相邻
		\end{itemize}

	\end{remark}


}{
	第一行,一个整数$t$,代表共有$t$组数据。

	第$2 \sim t+1$行,每行$14$个用空格分隔开的双字符字符串,代表一副手牌。手牌表示规则如下:

	\begin{itemize}
		\item 一个$1 \sim 9$的数字$x+$一个小写字母$b$,代表$x$饼(也叫$x$筒)。例如$2b$代表 \includegraphics[scale=0.5]{images/mahjong/2p.png}。
		\item 一个$1 \sim 9$的数字$x+$一个小写字母$t$,代表$x$条(也叫$x$索)。例如$8t$代表 \includegraphics[scale=0.5]{images/mahjong/8s.png}。
		\item 一个$1 \sim 9$的数字$x+$一个小写字母$w$,代表$x$万。例如$5w$代表 \includegraphics[scale=0.5]{images/mahjong/5m.png}。
		\item 一个$1 \sim 7$的数字$x+$一个小写字母$z$,从$1z \sim 7z$分别代表 \includegraphics[scale=0.5]{images/mahjong/1z.png}、\includegraphics[scale=0.5]{images/mahjong/2z.png}、\includegraphics[scale=0.5]{images/mahjong/3z.png}、\includegraphics[scale=0.5]{images/mahjong/4z.png}、\includegraphics[scale=0.5]{images/mahjong/5z.png}、\includegraphics[scale=0.5]{images/mahjong/6z.png}、\includegraphics[scale=0.5]{images/mahjong/7z.png}。
	\end{itemize}

	数据保证只会出现以上样式的牌。与正常的一副麻将不同,每张牌的出现次数\textbf{不限},例如可能出现14张白的情况,且这种情况是和牌。

}{
	共$t$行,第$i$行代表第$i$组数据的答案:若该组牌为和牌,输出\lstinline{Tsumo!};反之输出\lstinline{Waiting for Tsumo!}

}{
\begin{tabularx}{450pt}{X|X}
	\toprule
	输入样例                                  & 输出样例           \\
	\midrule
	2                                         & Waiting for Tsumo! \\
	1w 2w 3w 4b 5b 6b 7t 8t 9t 1b 1b 1z 2z 6z & Tsumo!             \\
	1w 2w 3w 4b 5b 6b 7t 8t 9t 1b 1b 2z 2z 2z &                    \\
	\bottomrule
\end{tabularx}

}
\section*{样例解释}

第一副手牌为

\begin{center}
	\includegraphics[scale=0.5]{images/mahjong/1m.png}\includegraphics[scale=0.5]{images/mahjong/2m.png}\includegraphics[scale=0.5]{images/mahjong/3m.png}\includegraphics[scale=0.5]{images/mahjong/4p.png}\includegraphics[scale=0.5]{images/mahjong/5p.png}\includegraphics[scale=0.5]{images/mahjong/6p.png}\includegraphics[scale=0.5]{images/mahjong/7s.png}\includegraphics[scale=0.5]{images/mahjong/8s.png}\includegraphics[scale=0.5]{images/mahjong/9s.png}\includegraphics[scale=0.5]{images/mahjong/1p.png}\includegraphics[scale=0.5]{images/mahjong/1p.png}\includegraphics[scale=0.5]{images/mahjong/1z.png}\includegraphics[scale=0.5]{images/mahjong/2z.png}\includegraphics[scale=0.5]{images/mahjong/6z.png}
\end{center}

第二副手牌为

\begin{center}
	\includegraphics[scale=0.5]{images/mahjong/1m.png}\includegraphics[scale=0.5]{images/mahjong/2m.png}\includegraphics[scale=0.5]{images/mahjong/3m.png}\includegraphics[scale=0.5]{images/mahjong/4p.png}\includegraphics[scale=0.5]{images/mahjong/5p.png}\includegraphics[scale=0.5]{images/mahjong/6p.png}\includegraphics[scale=0.5]{images/mahjong/7s.png}\includegraphics[scale=0.5]{images/mahjong/8s.png}\includegraphics[scale=0.5]{images/mahjong/9s.png}\includegraphics[scale=0.5]{images/mahjong/1p.png}\includegraphics[scale=0.5]{images/mahjong/1p.png}\includegraphics[scale=0.5]{images/mahjong/2z.png}\includegraphics[scale=0.5]{images/mahjong/2z.png}\includegraphics[scale=0.5]{images/mahjong/2z.png}
\end{center}
\clearpage

\makeproblem{F}{
	\begin{center}
		\includegraphics[scale=0.5]{images/God.png}
	\end{center}

	干员异客的攻击方式为:“攻击造成\textbf{法术伤害},且会在$4$个敌人间跳跃,每次跳跃伤害降低$15\%$并造成短暂停顿”。即:若敌人被击中时,当前攻击已经击中了 $i$ 个人,则该攻击会对其造成 $(85\%)^i$ 倍的\textbf{法术伤害}。

	法术伤害会受到敌人\textbf{法抗}的百分比削减,即若对一个\textbf{法抗}为$p\%$的敌人造成$s$点\textbf{法术伤害},该敌人仅会受到$s \times (1-p\%)$点\textbf{真实伤害}。

	假设敌人具有无限点血量,给出异客的攻击力和四个被击中的敌人的法抗,请你计算出异客一次攻击对所有敌人共造成了多少\textbf{真实伤害}。

	注意:因为敌人的血量为整数,所以对\textbf{每个敌人}打出的\textbf{真实伤害}都需要向下取整。因浮点误差影响,所输出答案与真实答案之间允许有$2$以内的误差。

}{
	第一行,一个整数$t(1 \leq t \leq 10^5)$,代表共有$t$组数据。

	第$2 \sim t+1$行,每行$5$个整数,分别代表异客的攻击力$A(1 \leq A \leq 1500)$和依次被击中的$4$个敌人的法抗$p_1,p_2,p_3,p_4(0 \leq p_i \leq 100)$

}{
	共$t$行,第$i$行代表第$i$组数据的答案

}{
	\begin{tabularx}{450pt}{X|X}
		\toprule
		输入样例         & 输出样例 \\
		\midrule
		3                & 2549     \\
		1000 20 20 20 20 & 3680     \\
		1500 10 20 30 40 & 1959     \\
		1001 25 33 37 70 &          \\
		\bottomrule
	\end{tabularx}

}

\end{document}